\documentclass[12pt, a4paper]{article}

\usepackage{amsmath}
\usepackage[utf8]{inputenc}
\usepackage[spanish]{babel} %Paquete de idioma
\usepackage[hidelinks]{hyperref}
\usepackage{graphicx}
\usepackage{float}
\usepackage{eso-pic}
\usepackage{lipsum}
\usepackage{transparent}
\usepackage{parskip}
%\usepackage[backend=biber, style=apa]{biblatex}

\graphicspath{{imagenes/}}

%%%%%%%%% ESTO PARA LA MARCA DE AGUA %%%%%%%%%%%%%%%%%%%%%%%%%%%%%%%


\AddToShipoutPicture{ 
    \put(410,380){
        \parbox[b][\paperheight]{\paperwidth}{%
            \vfill
            \
            {
            \transparent{1}
            \includegraphics[scale=0.5]{logo-ua.png}
            \vfill
            }
        }
    }
}
%%%%%%%%%%%%%%%%%%%%%%%%%%%%%%%%%%%%%%%%%%%%%%%%%%%%%%%%%%

\title{Detección de tuberias en imagenes submarinas} \\



\author{
Leopoldo Cadavid Piñero
}








\begin{document}

\maketitle
\newpage
\tableofcontents
\newpage

      


\section{Preprocesamiento}

El preprocesamiento de la imagen constará de 2 partes:

\begin{itemize}
    
    \item Aplicar una mascara para aumentar el contraste. Esto se hace para compensar
    la distorsión en los colores provocada por la propagación de la luz a lo largo de la imagen.
    


    \item Posteriormente, se aplica un CAHE (o \textit{contrast-limited adaptive histogram equalization} ).
    con el objetivo de redistribuir la luminiscencia en la imagen. 
    
    

\end{itemize}

La aplicación de los métodos da como resultado una imagen mán nítida y contrastada
 donde es más fácil aplicar los algoritmos para la segmentación y detección de las estructuras.


\section{Deteccion del objeto}

\subsection{Primera aproximación}

Siguiendo con los métodos referenciados, se propone el uso de un filtro sobel para obtener
las respuestas de gradiente, y añadir esta respuesta de gradiente a un vector de características,
con el objetivo de dividir la imagen según estas características en diferentes clusters. Para ello se pretende usar el algoritmo
k-means. 

\textbf{Primeros resultados:} tras aplicar el método descrito, las clusterización no arroja una segmentación prometedora.
Los fallos posibles:
\begin{itemize}
    \item Que los bordes no sean una características significativa a la hora de detectar
    el objeto. 

    \item Una implementación errónea del algoritmo a partir de la descripción. 
    \item Una representación errónea de los datos.
\end{itemize}


-Tras la comprobación de las distintas transformaciones aplicadas a la imagen, se observa que la conversión de escala rgb a escala 
de grises provoca una gran perdida de información, afectando esto al uso del algoritmo Sobel, pues se pierde información importante del cambio de 
gradiente en ciertos canales de color de la imagen.

A raíz de esto, se ha aplicado el filtro sobel a cada uno de los canales de la imagen por separado. Posteriormente, se calcula el gradiente medio de cada
canal y se añade cada uno al algoritmo k-means. De esta forma se ha conseguido mejorar sustancialmente la clusterización. Aún así, no es suficiente como para que 
haya separado la tuberia en la imagen de referencia. 

Lo proximo a tratar será la detección de lineas en cada una de las imagenes de gradiente, para ver si estas facilitan el proceso de segmentado





\section{Referencias}

https://journals.sagepub.com/doi/10.5772/60526

% \begin{figure}[H]
%     \centering
%     \includegraphics[scale=0.1]{tarea_1roscore.png}
%     \caption{Lanzamos el master}
%     \label{fig:tarea1roscore}
% \end{figure}



\end{document}