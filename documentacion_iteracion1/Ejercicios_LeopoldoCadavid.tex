\documentclass[12pt, a4paper]{article}

\usepackage{amsmath}
\usepackage[utf8]{inputenc}
\usepackage[spanish]{babel} %Paquete de idioma
\usepackage[hidelinks]{hyperref}
\usepackage{graphicx}
\usepackage{float}
\usepackage{eso-pic}
\usepackage{lipsum}
\usepackage{transparent}
\usepackage{parskip}
%\usepackage[backend=biber, style=apa]{biblatex}

\graphicspath{{imagenes/}}

%%%%%%%%% ESTO PARA LA MARCA DE AGUA %%%%%%%%%%%%%%%%%%%%%%%%%%%%%%%


\AddToShipoutPicture{ 
    \put(410,380){
        \parbox[b][\paperheight]{\paperwidth}{%
            \vfill
            \
            {
            \transparent{1}
            \includegraphics[scale=0.5]{logo-ua.png}
            \vfill
            }
        }
    }
}
%%%%%%%%%%%%%%%%%%%%%%%%%%%%%%%%%%%%%%%%%%%%%%%%%%%%%%%%%%

\title{Detección de tuberias en imagenes submarinas} \\



\author{
Leopoldo Cadavid Piñero
}








\begin{document}

\maketitle
\newpage
\tableofcontents
\newpage

      


\section{Preprocesamiento}

El preprocesamiento de la imagen constará de 2 partes:

\begin{itemize}
    
    \item Aplicar una mascara para aumentar el contraste. Esto se hace para compensar
    la distorsión en los colores provocada por la propagación de la luz a lo largo de la imagen.
    


    \item Posteriormente, se aplica un CAHE (o \textit{contrast-limited adaptive histogram equalization} ).
    con el objetivo de redistribuir la luminiscencia en la imagen. 
    
    

\end{itemize}




% \begin{figure}[H]
%     \centering
%     \includegraphics[scale=0.1]{tarea_1roscore.png}
%     \caption{Lanzamos el master}
%     \label{fig:tarea1roscore}
% \end{figure}



\end{document}