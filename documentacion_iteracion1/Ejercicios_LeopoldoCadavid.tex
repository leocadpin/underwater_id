\documentclass[12pt, a4paper]{article}

\usepackage{amsmath}
\usepackage[utf8]{inputenc}
\usepackage[english]{babel}
% \usepackage[spanish]{babel} %Paquete de idioma
\usepackage[hidelinks]{hyperref}
\usepackage{graphicx}
\usepackage{float}
\usepackage{eso-pic}
\usepackage{lipsum}
\usepackage{transparent}
\usepackage{parskip}
%\usepackage[backend=biber, style=apa]{biblatex}

\graphicspath{{images_doc/}}

%%%%%%%%% ESTO PARA LA MARCA DE AGUA %%%%%%%%%%%%%%%%%%%%%%%%%%%%%%%


\AddToShipoutPicture{ 
    \put(470,370){
        \parbox[b][\paperheight]{\paperwidth}{%
            \vfill
            \
            {
            \transparent{1}
            \includegraphics[scale=0.1]{uware_logo.png}
            % \includegraphics[scale=0.5]{logo-ua.png}
            \vfill
            }
        }
    }
}
%%%%%%%%%%%%%%%%%%%%%%%%%%%%%%%%%%%%%%%%%%%%%%%%%%%%%%%%%%

\title{Underwater pipe detector program   - Version 1} 


\author{
Leopoldo Cadavid Piñero
}








\begin{document}

\maketitle

\begin{figure}[H]
    \centering
    \includegraphics[scale=0.5]{images_doc/titel.png}
    
    \label{fig:repre}
\end{figure}

\newpage
\tableofcontents
\newpage

      
\section{Intro}

The main goal for this algorithm is to make possible human-made structures detection at underwater environments, making use of 
classic computer vision techniques.

Especifically, this first version of the algorithm is focused on pipes or similar structures. 

\section{Preprocessing}\label{ch:preprocesamiento}

First of all, is necessary to apply some preprocessing methods to get our images 
\textit{cleaner} before trying to get features from them. Preprocessing pipeline 
consist on 2 parts:

\begin{itemize}
    
    \item Apply a mask in order to increase contrast, decreasing L value in CIELAB channel space in our picture.
    This is done because we want to
    compensate colour distorsion result from ligth propagation along the picture. 
        

    \item Second, we use a CLAHE ( \textit{contrast-limited adaptive histogram equalization} ) in order to 
    \textbf{redistribute luminescence}.
    
    

\end{itemize}

La aplicación de los métodos da como resultado una imagen mán nítida y contrastada
 donde es más fácil aplicar los algoritmos para la segmentación y detección de las estructuras.


\section{Line detector}

\subsection{First approach}

Siguiendo con los métodos referenciados, se propone el uso de un filtro sobel para obtener
las respuestas de gradiente, y añadir esta respuesta de gradiente a un vector de características,
con el objetivo de dividir la imagen según estas características en diferentes clusters. Para ello se pretende usar el algoritmo
k-means. 

Based on referenced papers, the use of Sobel filter to get gradient response on different directions. Then 
this response would be used to create a \textbf{features vector}. The final step would be to classiffy objects 
using K-means algorithm with this feature array as entrance. 

% \textbf{Primeros resultados:} tras aplicar el método descrito, las clusterización no arroja una segmentación prometedora.
% Los fallos posibles:
% \begin{itemize}
%     \item Que los bordes no sean una características significativa a la hora de detectar
%     el objeto. 

%     \item Una implementación errónea del algoritmo a partir de la descripción. 
%     \item Una representación errónea de los datos.
% \end{itemize}

The results received from those transformations applied to our picture weren't successful. One of problems found, was an 
information loss, that appeared after grayscale conversion from RGB when we apply Sobel. Grayscale meshes all channels in one, ommitting 
part of the information. 


The solution implemmented for this issue has been 

As a result of this, the sobel filter has been applied to each of the image channels separately. Subsequently, the average gradient of each
channel and each is added to the k-means algorithm. In this way, clustering has been substantially improved. Still, it's not for making
an accurate segmentation.

At the moment, this approach has been abandoned, as long as it has not reported results of any kind. However, it could be 
interesting to come back to its study sometime in the future. 


\subsection{Second approach}



In this case, based on certain ideas discussed in the previous approximation,
we're trying get a different pipeline from the studied one.  The steps followed
in this case they have been the following:

\begin{itemize}
    
    \item Perform preprocessing, as described in \Ref{ch:preprocesamiento}.
    
    \begin{figure}[H]
        \centering
        \includegraphics[scale=0.3]{images_doc/preprocess.png}
        \caption{Before-After preprocessing}
        \label{fig:pprepre}
    \end{figure}

    \item Apply a bilateral filter to remove some background noise from images.
    
    \begin{figure}[H]
        \centering
        \includegraphics[scale=0.3]{images_doc/filtradobilat.png}
        \caption{bilateral filter}
        \label{fig:bilat}
    \end{figure}

    \item Apply \verb|PyrMeanShift()|, so that the image is segmented by colors.
    
    \begin{figure}[H]
        \centering
        \includegraphics[scale=0.3]{images_doc/meanshift.png}
        \caption{Meanshift segmentation}
        \label{fig:menas}
    \end{figure}

    \item Apply the algorithm \textbf{Sobel()}, to each of the channels of thepreprocessed  image
    RGB. This is done to get the gradient changes and is applied equally
    to the X and Y axes, performing a weighted sum of both in order to obtain the
    change in different directions.

    \begin{figure}[H]
        \centering
        \includegraphics[scale=0.3]{images_doc/gradientes.png}
        \caption{Sobel gradients splitted}
        \label{fig:Sobel}
    \end{figure}

    \item Obtain, from the gradients, edges in the images applying the
    algorithm \verb|Canny()|.

    \begin{figure}[H]
        \centering
        \includegraphics[scale=0.3]{images_doc/bordes.png}
        \caption{Edges with Canny}
        \label{fig:Canny}
    \end{figure}


    \item We use morphological filters to erode and dilate, trying to eliminate,
    circular elements that may disturb and then we amplify rectangular elements.

    \begin{figure}[H]
        \centering
        \includegraphics[scale=0.3]{images_doc/bordes_erode.png}
        \caption{Edges after morphological transformations}
        \label{fig:erodes}
    \end{figure}



    \item Having the edges filtered, the Hough transform is used to find lines in
    inside the image.

    \begin{figure}[H]
        \centering
        \includegraphics[scale=0.3]{images_doc/lineas_separada.png}
        \caption{Lines from Hough transform}
        \label{fig:jiug}
    \end{figure}
    
    \item From the 3 line vectors we have, we create a new one where we will join
    \textbf{all lines in the image}.
    
\end{itemize}

% \begin{figure}[H]
%     \centering
%     \includegraphics[scale=0.3]{sobel_canny_houghlines.png}
%     \caption{Lines obtained for each channel in one of the images studied}
%     \label{fig:sobelcanny}
% \end{figure}
\begin{figure}[H]
    \centering
    \includegraphics[scale=0.3]{images_doc/all_lines.png}
    \caption{Lines from Hough transform}
    \label{fig:jiiug}
\end{figure}
\section{Line filtering}


From the extracted lines, it seeks to dete lines that do not correspond to those belonging to the real pipe. For this, it will be analyzed
both the degree of parallelism and the distance between them.

The idea is to analyze in the vector of lines each pair of these in the following way:


\begin{itemize}

    \item The slope and the angle of each of these with respect to the front are calculated
    \item The angles are compared and, if they are below a certain threshold, the lines will be considered parallel
    \item The distances in X and Y of the initial points of each line are compared, and we are left with those that are below
    a threshold away
    \item Lines that meet all conditions will be saved in a vector of parallel lines
\end{itemize}


After filtering the lines, these are highlighted on the original image, applying certain morphological filters to improve their shape, and we finally obtain the
image with the mask applied on the structure.

\section{Subsection}



\subsection{Used Dataset}

Before showing the results, the images and video used to study the operation of the algorithm will be shown:

\begin{figure}[H]
    \includegraphics[width=.24\textwidth]{images_doc/img1.png}\hfill
    \includegraphics[width=.24\textwidth]{images_doc/img2.png}\hfill
    \includegraphics[width=.24\textwidth]{images_doc/img3.png}\hfill
    \includegraphics[width=.24\textwidth]{images_doc/img4.png}
    \\[\smallskipamount]
    \includegraphics[width=.24\textwidth]{images_doc/img5.png}\hfill
    \includegraphics[width=.24\textwidth]{images_doc/img6.png}\hfill
    \includegraphics[width=.24\textwidth]{images_doc/img7.png}\hfill
    \includegraphics[width=.24\textwidth]{images_doc/img8.png}
   
    \caption{Some images}\label{fig:foobar}
\end{figure}

El video utlizado ha sido el siguiente \href{https://youtu.be/L0Ev5R3fBEc}{https://youtu.be/L0Ev5R3fBEc}

\newpage
\begin{figure}[H]
    \includegraphics[scale=0.22]{images_doc/img1_results.png}
    \caption{res1}\label{fig:im1r}
\end{figure}

\newpage
\begin{figure}[H]
    \includegraphics[scale=0.22]{images_doc/img2_results.png}
    \caption{res1}\label{fig:im2r}
\end{figure}
\newpage
\begin{figure}[H]
    \includegraphics[scale=0.22]{images_doc/img3_results.png}
    \caption{res1}\label{fig:im3r}
\end{figure}

\newpage
\begin{figure}[H]
    \includegraphics[scale=0.22]{images_doc/img4_results.png}
    \caption{res1}\label{fig:im4r}
\end{figure}
\begin{figure}[H]
    \includegraphics[scale=0.22]{images_doc/img5_results.png}
    \caption{res1}\label{fig:im5r}
\end{figure}

\newpage
\begin{figure}[H]
    \includegraphics[scale=0.22]{images_doc/img6_results.png}
    \caption{res1}\label{fig:im6r}
\end{figure}
\newpage
\begin{figure}[H]
    \includegraphics[scale=0.22]{images_doc/img7_results.png}
    \caption{res1}\label{fig:im7r}
\end{figure}

\newpage
\begin{figure}[H]
    \includegraphics[scale=0.22]{images_doc/img8_results.png}
    \caption{res1}\label{fig:im8r}
\end{figure}

\subsection{Possible improvements}


\begin{itemize}

    \item Tune or delete some esteps in order to increasing computing velocity. Analyze each algorithm time-cost vs its
    ultility inside our pipeline. 
    
    \item Make line detection more sophisticated in order to avoid information losses or fake lines. Study new algorithms or 
    methrics for the used ones. 
    
    \item Implement as a ROS node and make performance tests with more images.

\end{itemize}



\section{Referencias}

https://journals.sagepub.com/doi/10.5772/60526

% \begin{figure}[H]
%     \centering
%     \includegraphics[scale=0.1]{tarea_1roscore.png}
%     \caption{Lanzamos el master}
%     \label{fig:tarea1roscore}
% \end{figure}



\end{document}